\documentclass[12pt,a4paper,twoside]{ipb}

% comentar caso seja uma disseração de mestrado
% \usepackage{projei}

\usepackage[portuguese]{babel}
\graphicspath{{./images/}}
\usepackage{listings} % incluir listagens
\usepackage{url} % typeset URL's
\usepackage[colorlinks=true,
                    urlcolor=black, %blue
                    linkcolor=black,
                    citecolor=black, %cor das citações
                    bookmarks=true,
                    pdfstartview=FitH]{hyperref}

% Pode ser usado biblatex
\usepackage[style=ieee,backend=biber]{biblatex}
\addbibresource{lib/refs.bib}

\usepackage{lipsum}

\usepackage{tabulary}

%%%%%%%%%%%%%%%%%%%%%%%%%

\title{Desenvolvimento de um módulo de gestão inteligente e sustentável da construção civil, baseado em Business Analytics}

\author{Vinícius Nascimento Silva - a62860}
%\authnum{1}

\supervisor{Prof. Paulo Alexandre Vara Alves}
\cosupervisor{Prof. Eduardo Cunha Campos}

% Para definir o ano letivo
\courseyear{2024-2025}

% Para nao mostrar a lista de tabelas
%\tablespagefalse

\makeglossaries
\loadglsentries{acronym}


%http://tex.stackexchange.com/questions/59572/custom-page-numbering-for-appendix
\usepackage{etoolbox}
\usepackage{parskip}


\begin{document}
	
% Coloca a capa, primeira pagina e outros

\beforepreface

%\cleardoublepage

\prefacesection*{Dedicatória}
%\thispagestyle{empty}

%\lipsum[1]

(Facultativo) Dedico este trabalho a ...

%\vfill
%\pagebreak
%\mbox{}
%\vfill
%\pagebreak

%\cleardoublepage

\prefacesection*{Agradecimentos}
%\thispagestyle{empty}

%\lipsum[1]

Agradeço a meus pais e meus professores.

%\vfill
%\pagebreak
%\mbox{}
%\vfill
%\pagebreak


%\cleardoublepage

\prefacesection*{Resumo}
%\thispagestyle{empty}

O projeto propõe o desenvolvimento e a integração de um módulo de \textit{Business Analytics} em uma plataforma de gestão voltada para pequenas e médias empresas (PMEs) do setor da construção civil. Este trabalho tem como objetivo complementar uma ferramenta de \textit{Enterprise Resource Planning} (ERP) já existente, fornecendo correções, melhorias e módulos adicionais. A metodologia adotada baseou-se no framework \textit{Scrum}, com \textit{Sprints} quinzenais e alocação de tarefas dimensionadas para um único desenvolvedor. Os painéis interativos e relatórios dinâmicos propostos devem possibilitar uma visão integrada da operação e apoiar a tomada de decisões estratégicas. Espera-se que o projeto contribua para a transformação digital do setor, ampliando a competitividade e a capacidade de gestão da empresa envolvida.

\mbox{}\linebreak
\noindent {\bf Palavras-chave:} Análise de Dados, Análise de Negócio, Gestão da Construção, Plataforma Web

%\vfill
%\pagebreak
%\mbox{}
%\vfill
%\pagebreak

%\cleardoublepage

\prefacesection*{Abstract}
%\thispagestyle{empty}

This project proposes the development and integration of a Business Analytics module into a management platform aimed at small and medium-sized enterprises (SMEs) in the construction sector. This work aims to complement an existing Enterprise Resource Planning (ERP) tool, providing corrections, improvements, and additional modules. The methodology adopted was based on the Scrum framework, with bi-weekly sprints and task allocation tailored to a single developer. The proposed interactive dashboards and dynamic reports should enable an integrated view of operations and support strategic decision-making. It is expected that the project will contribute to the digital transformation of the sector, increasing the competitiveness and management capacity of the companies involved.

\mbox{}\linebreak
\noindent {\bf Keywords:} Data Analysis, Business Analytics, Construction Management, Web Platform

%\vfill
%\pagebreak
%\mbox{}
%\vfill
%%\pagebreak

% Coloca indices
\afterpreface
%\cleardoublepage
%\printglossary[type=\acronymtype,title={Acrónimos}]
\printglossary[type=\acronymtype,title={Siglas}]

\bodystart


% Capitulos do documento
\cleardoublepage
\chapter{Introdução}\label{cap:intro}

No contexto de empresas de construção civil no distrito de Bragança em Portugal, existe a empresa \gls{ALN}, uma \gls{PME} da região especializada em execução de obras públicas e privadas. Nessa esfera, gerir recursos e processos de forma eficiente é fundamental para garantir competitividade nesse ramo de negócio. Para isso, existem \textit{softwares} com foco em \gls{PRE}.

Entre 2023 e 2024, uma nova parceria entre a \gls{ALN} e o \gls{IPB} resultou no desenvolvimento de uma nova plataforma web com foco em \gls{PRE}. Esta nova solução foi baseada em um projeto de 2013, entretanto, a nova versão não foi oficialmente para produção por não possuir boa parte das funcionalidades essenciais que a antiga ferramenta era capaz de cobrir. Frente a necessidade de substituir a plataforma obsoleta por tecnologias modernas e complementar o produto surge uma terceira parceria entre os anos de 2024 e 2025.

O objetivo geral desta nova etapa está em corrigir, completar funcionalidades ausentes e integrar novos módulos de \textit{business analytics} de modo a suportar uma gestão inteligente e sustentável da construção. Já os objetivos específicos incluem desenvolvimento do Módulo de Gestão de Obras, do Módulo de Parametrização, do Módulo de Registo de Ponto e do Módulo de Gestão Inteligente e Sustentável da Construção.

Entre as ferramentas neste contexto de implementação estão Primavera Contruction, PHC CS Construção e Eticadata Obras. Todas oferecem funcionalidades de gestão orçamental, contabilística e acompanhamento de projetos. Existem também alternativas mais robustas como SAP Business One, Microsoft Dynamics 365, Autodesk Construction Cloud, Procore e Buildertrend que para além dos recursos básicos trazem colaboração em tempo real, metodologias \gls{BIM} e integração com análises preditivas. Embora estas soluções apresentem maturidade, elas implicam custos elevados de licenciamento, necessidade de adaptação à realidade local ou ausência de módulos específicos, o que justifica o desenvolvimento de uma plataforma ajustada à realidade da \gls{ALN}.

Os próximos capítulos desta dissertação seguem modelo estruturado. O Estado da Arte será apresentado no Capítulo \ref{cap:conceptual} e será feita comparações com ferramentas pré-disponíveis no mercado nesse setor. Já no Capítulo \ref{cap:metodology}, serão explorados os detalhes sobre as correções necessárias, os módulos que faltavam e as histórias de usuário dos novos recursos. Em seguida, no Capítulo \ref{cap:development}, são apresentados o desenvolvimento do projeto, as mudanças que foram necessárias no banco de dados, alterações do frontend e backend. No Capítulo \ref{cap:test}, são detalhados os resultados obtidos neste projeto. Por fim, no Capítulo \ref{cap:conclusions}, estão as conclusões do presente trabalho.
\cleardoublepage
\chapter{Estado da Arte}\label{cap:art}

As técnicas de análise de dados aplicadas nas empresas são classificadas em cinco tipos: Descritiva, Diagnóstica, Preditiva, Prescritiva e Cognitiva. A classificação Descritiva explica o ocorrido, através de tecnologias de \gls{BI}. Já a Diagnóstica busca explicar a causa raiz, enquanto a Preditiva procura antecipar eventos futuros. Em sequência, a Prescritiva define a intervenção a ser feita. Por fim, a Cognitiva visa extrair conexões entre os dados \cite{Kambhampaty2021}. A integração dessas análises permite que as empresas antecipem tendências futuras, identifiquem riscos potenciais e tomem decisões baseadas em dados \cite{Agu2024}.

O setor da construção civil produz grandes volumes de dados heterogêneos e dinâmicos, fragmentados ao longo do ciclo de vida de uma obra \cite{Rodrigues2022}. No entanto, a indústria da construção civil é caracterizada pela lentidão na adoção de novas tecnologias em comparação com outros setores \cite{Lopes2020}, bem como por uma barreira cultural potencializada pela falta de mão de obra qualificada \cite{Aykanat2025}. Somado a isso, as fontes de dados incluem \textit{datasets} de simulação, experimentais, organizacionais, de monitorização de projetos e fontes não estruturadas, como redes sociais e relatórios \cite{Li2023}. A adoção de \gls{BI} no setor é aplicada para diversos fins, como estimativa de custos e orçamentos, previsão de atrasos, previsão de consumo de recursos e gestão da produtividade \cite{Ngo2020}.

Os principais problemas que persistem nos projetos de construção civil incluem a falta de abordagens unificadas de gerenciamento de dados, ineficiências no planejamento, coordenação e comunicação \cite{Wattuhewa2023}. Adicionalmente, erros de projeto levam a aproximadamente 33\% do desperdício de materiais e recursos \cite{Ram2019}. Por outro lado, isso abre espaço para aplicações de outras tecnologias, como sensores \gls{IoT}, realidade virtual, realidade aumentada, inteligência artificial, aprendizado de máquina e impressão 3D \cite{Kumar2023}. 

No escopo de \textit{Big Data}, existem três conceitos-chave: velocidade, volume e variedade \cite{Ochuba2024}. Para lidar com isso, o foco central é o \gls{BIA}. Este termo abrange metodologias utilizadas para coletar, analisar e disseminar dados, de forma a favorecer tomadas de decisão estratégicas e fundamentadas \cite{Bany2022}. Estão incluídas ferramentas \gls{ETL}, \gls{DW}, tecnologia \gls{OLAP}, \gls{DM} e relatórios textuais ou gráficos \cite{AlKhatib2021} \cite{AlOkaily2022}. Em outras palavras, cada uma delas cobre objetivos específicos. O primeiro é o \gls{ETL}, que representa a coleta, transformação e carregamento dos dados no sistema \cite{Bany2022}. Cabe ressaltar que digitalizar documentos faz parte deste grupo. Outro pilar é o \gls{DW}, repositório que armazena o grande volume de dados \cite{Daissaoui2020} e é otimizado para consultas ao acervo \cite{AlOkaily2022}. Já no contexto de \gls{DM}, esta técnica busca classificar, agrupar, associar e prever dados \cite{Ericsson2022}. Por outra frente, a abordagem \gls{OLAP} permite visualizações rápidas de dados por diversos ângulos e navegação histórica ou pelos relacionamentos entre eles \cite{Kambhampaty2021}. Já os relatórios, geralmente construídos como painés, apresentam em uma única tela os pontos mais importantes para a tomada de decisão \cite{Rodrigues2022}.

Em razão das características únicas dos projetos de construção civil, grande parte dos \textit{softwares} exige personalização, seja pela possibilidade de customização de alternativas existentes, seja pelo desenvolvimento sob demanda de aplicações específicas. No mercado português e mundial existem diversas soluções orientadas à gestão de recursos na construção civil (\gls{PRE}). Cada uma dessas ferramentas apresenta diferentes graus de maturidade, funcionalidades e custos, sendo por vezes mais ajustadas a grandes empresas do que a \gls{PME}. A Tabela \ref{tab:comparacao_softwares} apresenta uma comparação entre algumas das soluções mais relevantes, destacando vantagens e limitações. Esta análise permite contextualizar este estudo e reforça a necessidade de uma solução adaptada à realidade local da \gls{ALN}.

\begin{table}[H]
    \centering
    \caption{Comparação de soluções de software de PRE no setor da construção}
    \label{tab:comparacao_softwares}
    \begin{tabular}{|p{3cm}|p{5cm}|p{5cm}|}
    \hline
\textbf{Ferramenta} & \textbf{Vantagens} & \textbf{Limitações} \\ \hline

Primavera Construction & 
Integra contabilidade, gestão de obras e RH; forte presença em Portugal; adequado para PME. & 
Licenciamento relativamente caro; curva de aprendizagem para utilizadores menos experientes. \\ \hline

PHC CS Construção & 
Gestão de contratos, empreitadas e faturação; interface amigável; solução nacional. & 
Menos flexível para integração com plataformas externas; custo médio. \\ \hline

ETICADATA Obras & 
Boa adaptação à realidade portuguesa; módulos de orçamento e acompanhamento de obra. & 
Menor suporte para análises avançadas e integração com \gls{BI}. \\ \hline

SAP Business One (Construção) & 
Altamente robusto; integra finanças, compras, logística e construção; escalável para grandes empresas. & 
Custos muito elevados; elevado tempo de implementação; pouco ajustado a PME. \\ \hline

Microsoft Dynamics 365 (+ Add-ons) & 
Flexível; integração nativa com Power BI; forte capacidade analítica e modular. & 
Licenciamento elevado; necessidade de customização para o setor da construção. \\ \hline

Autodesk Construction Cloud & 
Colaboração em tempo real; suporte BIM; integração de documentação e cronogramas. & 
Mais orientado para gestão de projetos do que para gestão financeira; custos de subscrição. \\ \hline

Procore & 
Foco em gestão de projetos; comunicação eficiente entre equipas; solução \textit{cloud}. & 
Mais popular em mercados internacionais; custo elevado; pouca adaptação ao contexto português. \\ \hline

Buildertrend & 
Orientado a PME; interface intuitiva; acesso em nuvem; foco em orçamentos e cronogramas. & 
Funcionalidades financeiras limitadas; menor presença na Europa; dependência da \textit{cloud}. \\ \hline

\end{tabular}
\end{table}

\cleardoublepage
\chapter{Abordagem/Análise/Modelação}\label{cap:metodology}

Neste capítulo espera-se uma descrição detalhada do problema e da proposta de solução. 

No caso de projetos de desenvolvimento de software, deverá deitar-se mão dos conceitos e ferramentas de Análise/Modelação estudadas durante o curso (por exemplo, diagramas UML ou outra linguagem gráfica). Deve-se indicar explicitamente as tarefas a desempenhar pelo sistema, e os atores que interagem com o mesmo. A descrição deve ter suficiente detalhes  para perceber as dificuldades associadas à resolução do problema.

\cleardoublepage
\chapter{Desenvolvimento/Implementação}\label{cap:development}

Neste capítulo é descrito o trabalho de implementação, salientando os pontos mais relevantes da mesma, dificuldades  encontradas ou soluções técnicas inovadoras desenvolvidas ou aplicadas.  Em particular, se foi usado código desenvolvido por terceiros (por exemplo, código {\it open-source}), deve ser facilmente distinguível quais as funcionalidades originais do mesmo e o que foi necessário implementar para obter as funcionalidades desejadas.

\cleardoublepage
\chapter{Discussão}\label{cap:test}

Após as implementações apresentadas no capítulo anterior, os grupos pré-definidos \textbf{Equipamentos e Serviços} e \textbf{Obras} passaram a apresentar a organização demonstrada nas Figuras \ref{equipamentos-e-servicos} e \ref{obras}. 

\begin{figure}[H]
    \centering 
    \includegraphics[width=0.4\textwidth]{images/equipamentos-e-servicos.png} 
    \caption{Equipamentos \& Serviços - Versão Final} 
    \label{equipamentos-e-servicos}
\end{figure}

\begin{figure}[H]
    \centering 
    \includegraphics[width=0.4\textwidth]{images/obras.png} 
    \caption{Obras - Versão Final} 
    \label{obras}
\end{figure}

O mesmo ocorre com os agrupamentos \textbf{Outros Custos} e \textbf{Recursos Humanos}, conforme ilustrado nas Figuras \ref{outros-custos} e \ref{recursos-humanos}. Cabe ressaltar que a página \textit{Serviços e Consumíveis} foi renomeada para \textit{Fatura Serviços e Consumíveis}, de modo a melhorar a distinção nominal em relação à página homônima presente no grupo Equipamentos e Serviços.

\begin{figure}[H]
    \centering 
    \includegraphics[width=0.4\textwidth]{images/outros-custos.png} 
    \caption{Outros Custos - Versão Final} 
    \label{outros-custos}
\end{figure}

\begin{figure}[H]
    \centering 
    \includegraphics[width=0.4\textwidth]{images/recursos-humanos.png} 
    \caption{Recursos Humanos - Versão Final} 
    \label{recursos-humanos}
\end{figure}

Os novos grupos \textbf{Amortizações Fiscais}, \textbf{Entidades} e \textbf{Parametrizações}, já mencionados no capítulo anterior, ficaram organizados como apresentado nas Figuras \ref{amortizacoes-fiscais}, \ref{entidades} e \ref{parametrizacoes}.

\begin{figure}[H]
    \centering 
    \includegraphics[width=0.4\textwidth]{images/amortizacoes-fiscais.png} 
    \caption{Amortizações Fiscais - Versão Final} 
    \label{amortizacoes-fiscais}
\end{figure}

\begin{figure}[H]
    \centering 
    \includegraphics[width=0.4\textwidth]{images/entidades.png} 
    \caption{Entidades - Versão Final} 
    \label{entidades}
\end{figure}

\begin{figure}[H]
    \centering 
    \includegraphics[width=0.4\textwidth]{images/parametrizacoes.png} 
    \caption{Parametrizações - Versão Final} 
    \label{parametrizacoes}
\end{figure}

Em relação aos testes de disponibilidade do sistema, registraram-se duas ocasiões em que o servidor ficou indisponível. A primeira ocorreu devido a uma atualização automática do \textit{Windows 11}, situação que não pode ser contornada pelos mecanismos de inicialização configurados no Sistema Operacional, exigindo intervenção humana. O segundo caso foi causado pelo modo de hibernação automática do \textit{Windows 11}; entretanto, essa situação pôde ser resolvida mediante ajustes nas configurações de energia da máquina.

\cleardoublepage
\chapter{Conclusões}\label{cap:conclusions}

À medida que as correções foram implementadas e novas versões foram disponibilizadas ao cliente, o número de novas tarefas relacionadas a ajustes de funcionamento começou a diminuir, conforme esperado. Esse cenário favorece que, nas próximas \textit{sprints}, as atividades voltadas aos módulos de \gls{BI} passem a receber maior prioridade.

O principal ponto fraco desta primeira etapa foi o fato de que o desenvolvimento da ferramenta como um todo ocorreu de maneira relativamente distante do cliente, o que contribuiu para o surgimento de diversas tarefas de retrabalho. Outro aspecto negativo consiste no fato de que algumas funcionalidades implementadas não se mostraram tão relevantes no contexto inicial e, portanto, não necessariamente precisariam ter sido desenvolvidas antes do lançamento em produção. Uma versão reduzida da ferramenta poderia ter sido disponibilizada previamente, embora a definição do grau de importância de cada página tenha se mostrado um desafio.

No que diz respeito ao sistema como um todo, observa-se que o produto apresenta um nível maior de estabilidade e está mais alinhado às expectativas da empresa \gls{ALN}. Contudo, ainda faltam páginas fundamentais para que os dados já inseridos possam efetivamente agregar valor ao processo de tomada de decisão por parte da gestão. Sem a presença de dashboards, o sistema, por si só, constitui um amplo repositório de informações, complementado apenas por alguns indicadores presentes na página de Execução Orçamental.

Como trabalho futuro, cabe implementar os \textit{mockups} propostos, bem como criar uma página para registro de capítulos para as obras. Os capítulos precisam possuir datas esperadas e reais além de um campo para registro do orçamento. Isso não existia na modelagem anterior. Somado a isso, será necessário identificar, registrar e resolver novas tarefas que podem surgir nas próximas semanas por \textit{feedback} de uso da ferramenta na empresa.

%% estilo de referências. outros valores posíveis são 'plain' e 'abbrv' apalike
%\bibliographystyle{plain}
%% listagem de referências
%\bibliography{lib/refs}

%  Caso seja usado biblatex
\printbibliography


% Apêndices
\appendix

%http://tex.stackexchange.com/questions/59572/custom-page-numbering-for-appendix
\pretocmd{\chapter}{%
	\clearpage
	\pagenumbering{arabic}%
	\renewcommand*{\thepage}{\thechapter\arabic{page}}%
}{}{}

\chapter{Proposta Original do Projeto}
\label{apendice1}

\begin{figure}
%\centering 
\hspace{-12ex}
\includegraphics{etc/PropostaProjeto.pdf}
\end{figure}
\chapter{Outro(s) Apêndice(s)}
\label{apendice2}

Listagens de código fonte, texto/imagens produzidos por testes complementares, etc.


\end{document}
