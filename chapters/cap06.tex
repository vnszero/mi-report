\chapter{Conclusões}\label{cap:conclusions}

À medida que as correções foram implementadas e novas versões foram disponibilizadas ao cliente, o número de novas tarefas relacionadas a ajustes de funcionamento começou a diminuir, conforme esperado. Esse cenário favorece que, nas próximas \textit{sprints}, as atividades voltadas aos módulos de \gls{BI} passem a receber maior prioridade.

O principal ponto fraco desta primeira etapa foi o fato de que o desenvolvimento da ferramenta como um todo ocorreu de maneira relativamente distante do cliente, o que contribuiu para o surgimento de diversas tarefas de retrabalho. Outro aspecto negativo consiste no fato de que algumas funcionalidades implementadas não se mostraram tão relevantes no contexto inicial e, portanto, não necessariamente precisariam ter sido desenvolvidas antes do lançamento em produção. Uma versão reduzida da ferramenta poderia ter sido disponibilizada previamente, embora a definição do grau de importância de cada página tenha se mostrado um desafio.

No que diz respeito ao sistema como um todo, observa-se que o produto apresenta um nível maior de estabilidade e está mais alinhado às expectativas da empresa \gls{ALN}. Contudo, ainda faltam páginas fundamentais para que os dados já inseridos possam efetivamente agregar valor ao processo de tomada de decisão por parte da gestão. Sem a presença de dashboards, o sistema, por si só, constitui um amplo repositório de informações, complementado apenas por alguns indicadores presentes na página de Execução Orçamental.

Como trabalho futuro, cabe implementar os \textit{mockups} propostos, bem como criar uma página para registro de capítulos para as obras. Os capítulos precisam possuir datas esperadas e reais além de um campo para registro do orçamento. Isso não existia na modelagem anterior. Somado a isso, será necessário identificar, registrar e resolver novas tarefas que podem surgir nas próximas semanas por \textit{feedback} de uso da ferramenta na empresa.