\chapter{Conclusões}\label{cap:conclusions}

A medida que as correções foram implementadas e novas versões disponibilizadas ao cliente. O número de novas tarefas para atender expectativas de funcionamento começaram a diminuir como esperado. Isso eventualmente favorece que as tarefas relacionadas com os módulos de \gls{BI} sejam os objetivos dos próximos \textit{sprints}.

O ponto fraco da primeira etapa, é que o desenvolvimento da ferramenta como um todo aconteceu distante do cliente e isso favoreceu que muitas tarefas de retrabalho surgissem. Outro ponto negativo é que parte das ferramentas implementadas não parece ser tão relevante e eventualmente não precisavam ser implementadas antes do lançamento em produção. Uma versão menor da ferramenta poderia ter sido lançada antes, mas era um pouco difícil pontuar a importância de cada página.

Em relação ao sistema como um todo, é visível que o produto parece mais estável e alinhado com as expectativas da empresa \gls{ALN}. Entretanto, ainda faltam páginas fundamentais para que os dados até então preenchidos agregem valor na tomada de decisão por parte da gestão da empresa. Sem os dashboards, o sistema por si só é um grande acervo apenas com alguns poucos indicadores na página de Execução Orçamental.