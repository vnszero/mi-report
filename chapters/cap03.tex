\chapter{Metodologia}\label{cap:metodology}

Neste capítulo, descreve-se a metodologia seguida no decorrer do projeto, apresentando as etapas, técnicas e ferramentas que sustentaram a construção da solução. Por se tratar do desenvolvimento de um software, a metodologia adotada centra-se nos processos de análise, implementação e validação do sistema, baseando-se em práticas consolidadas da engenharia de software.

\section{Ponto de Partida}

O presente trabalho dá continuidade aos esforços da dissertação \textit{Desenvolvimento de uma Plataforma de Gestão da Construção}, de Luiz Oliveira \cite{Oliveira2024}. Assim, parte do sistema já se encontrava modelada e implementada. Havia 4 grupos com 18 ferramentas estabelecidas: Equipamentos \& Serviços, Obras, Outros Custos e Recursos Humanos. O primeiro grupo era constituído 2 ferramentas: \textbf{Material} e \textbf{Consumível/Serviço}, conforme ilustrado na Figura \ref{2024-equipamentos-e-servicos}. Já o grupo Obras era composto por 8 páginas: \textbf{Obra}, \textbf{Execução Orçamental}, \textbf{Presenças}, \textbf{Preço Materiais}, \textbf{Materiais por Obra}, \textbf{Fatura Material}, \textbf{Pedido Cotação} e \textbf{Fatura Cliente}, conforme apresentado na Figura \ref{2024-obras}.

\begin{figure}[H]
    \centering
    \includegraphics[width=0.4\textwidth]{images/2024-equipamentos-e-servicos.jpeg}
    \caption{Equipamentos \& Serviços - Ponto de Partida}
    \label{2024-equipamentos-e-servicos}
\end{figure}

\begin{figure}[H]
    \centering
    \includegraphics[width=0.4\textwidth]{images/2024-obras.jpeg}
    \caption{Obras - Ponto de Partida}
    \label{2024-obras}
\end{figure}

Em seguida, o grupo Outros Custos integrado por 7 ferramentas: \textbf{IUC}, \textbf{Inspeção Periódica}, \textbf{Leasing Viaturas}, \textbf{Inspeção Periódica}, \textbf{Seguro Automóvel}, \textbf{Seguro Trabalho} e \textbf{Consumíveis e Serviços}, conforme ilustrado na Figura \ref{2024-outros-custos}. Por fim, Recursos Humanos possuía apenas 1 página: \textbf{Colaboradores}, apresentada na Figura \ref{2024-recursos-humanos}.

\begin{figure}[H]
    \centering
    \includegraphics[width=0.4\textwidth]{images/2024-outros-custos.jpeg}
    \caption{Outros Custos - Ponto de Partida}
    \label{2024-outros-custos}
\end{figure}

\begin{figure}[H]
    \centering
    \includegraphics[width=0.4\textwidth]{images/2024-recursos-humanos.jpeg}
    \caption{Recursos Humanos - Ponto de Partida}
    \label{2024-recursos-humanos}
\end{figure}

Dispor de um ponto inicial fornecia uma direção clara para o que deveria ser a implementação das páginas futuras, além de estabelecer previamente a arquitetura do projeto tanto no \textit{frontend} quanto no \textit{backend}. Por outro lado, essa base também implicava a herança de fragilidades estruturais e, embora houvesse código inicial, ele não correspondia a ferramentas prontas para uso. Ademais, para iniciar os testes com o cliente e garantir a cobertura mínima dos casos de uso, era necessário implementar outras dezessete ferramentas ainda ausentes. O primeiro passo consistiu em identificar bugs e resolver problemas nas ferramentas supostamente concluídas; em seguida, completar as funcionalidades em falta e, somente após isso, avançar para os novos módulos de \gls{BI}.

\section{Análise de Requisitos}

Em reunião com o cliente, foi possível estabelecer os requisitos funcionais dos novos módulos e identificar os dados realmente relevantes para a tomada de decisão dos gestores. De forma geral, o novo módulo deveria constituir um painel que reunisse as principais informações do negócio, permitindo ao gestor compreender o panorama geral em uma única tela. No contexto global, destacaram-se: obras ativas por ano, obras com maior consumo de materiais, classificação de materiais mais utilizados e classificação de fornecedores mais recorrentes. No contexto específico de uma obra, consideraram-se: tempo de extrapolação do previsto e comparação entre orçamento e custo real.

Identificou-se ainda a necessidade de criar 2 ferramentas adicionais: Capítulo no grupo de Parametrizações e Capítulos por Obra no grupo de Obras. A responsabilidade de definir os nomes e descrição dos possíveis capítulos enquadra-se no grupo de Parametrizações, enquanto que a responsabilidade de atribuir capítulos a obras em caráter de tempo e orçamento encaixa no grupo de Obras. A Figura \ref{diagrama-classe-capitulos} traz o diagrama de classe que ilustra o relacionamento entre esses conceitos.

\begin{figure}[H]
    \centering
    \includegraphics[width=0.4\textwidth]{images/diagrama-classes-capitulos.png}
    \caption{Diagrama de Classe para Capítulos}
    \label{diagrama-classe-capitulos}
\end{figure}

\section{Mockups}

Após a análise de requisitos, procedeu-se à criação de \textit{mockups} de interface para validar o fluxo de interação e a organização visual. Ferramentas de prototipagem, como o \textit{Figma}, foram utilizadas para construir protótipos de baixa e média fidelidade. A Figura \ref{mockup-indicadores-globais} traz a modelagem da página de Indicadores Globais. Nela, o usuário é capaz de filtrar por um intervalo de datas e assim obter informações sobre obras ativas nesse período de tempo, total de obras finalizadas até aquela data final, clientes com mais obras e total de obras naquele cliente. Além disso, a ferramenta apresenta a classificação de materiais, fornecedores, clientes e obras em critério custo real.

\begin{figure}[H]
    \centering
    \includegraphics[width=0.9\textwidth]{images/mockup-indicadores-globais.png}
    \caption{Indicadores Globais}
    \label{mockup-indicadores-globais}
\end{figure}

O próximo \textit{mockup}, conforme Figura \ref{mockup-indicadores-especificos}, exibe que os indicadores específicos permitem ao usuário escolher intervalo de datas, obra e um capítulo determinado. Assim, é possível compreender extrapolações temporais em relação às datas de início e fim daquela obra, bem como os custos de mão de obra, materiais e totais.

\begin{figure}[H]
    \centering
    \includegraphics[width=0.9\textwidth]{images/mockup-indicadores-especificos.png}
    \caption{Indicadores Específicos}
    \label{mockup-indicadores-especificos}
\end{figure}

Os \textit{mockups} permitiram antecipar problemas de usabilidade, ajustar elementos de navegação e assegurar que a solução final estivesse alinhada com as expectativas do cliente. Essas representações também serviram como referência direta para as etapas de implementação.

\section{Arquitetura Adotada}

A arquitetura do sistema utiliza, no \textit{backend}, a linguagem C\# com o \textit{framework} \textit{.NET}, empregando o \textit{Entity Framework Core} para interação com o banco de dados relacional \textit{MySQL}. No \textit{frontend}, utilizou-se a linguagem TypeScript com o \textit{framework} \textit{React} e a biblioteca \textit{Ant Design}. Essa escolha favoreceu o desenvolvimento, possibilitando o reaproveitamento significativo da estrutura legada da solução anterior, originalmente desenvolvida em 2013.

\section{Metodologia Ágil de Desenvolvimento}

A metodologia de trabalho adotada baseia-se em uma abordagem ágil, utilizando Kanban e ciclos iterativos quinzenais de tarefas. Essa abordagem permitiu gerir os esforços de forma incremental, priorizar funcionalidades essenciais e garantir rápida adaptação a novos requisitos \cite{Valente2020}. Foram definidas 85 tarefas no módulo \textit{Projects} do \textit{GitHub}, abrangendo novas funcionalidades, melhorias, correções e refatorações em todas as frentes do \textit{software}.

Complementarmente, o uso do Kanban possibilitou a visualização contínua do fluxo de trabalho, promovendo maior transparência, controlo do progresso e identificação precoce de gargalos no desenvolvimento. As tarefas foram organizadas em colunas representativas dos estados de execução, permitindo o acompanhamento sistemático desde a conceção até à entrega. Os ciclos iterativos quinzenais favoreceram momentos regulares de planeamento, revisão e ajuste, incentivando a melhoria contínua do processo, a validação incremental das funcionalidades implementadas e o alinhamento constante entre os objetivos técnicos e os requisitos do projeto.

\section{Validação}

A validação do produto ocorreu de duas formas ao longo do projeto. Na primeira fase, as validações foram conduzidas junto ao professor orientador, em reuniões quinzenais. Após atingir um nível de maturidade suficiente para testes em produção, a validação passou a ocorrer de forma contínua com o cliente, por meio de testes de disponibilidade e verificação semanal de funcionalidades. A revisão das novas telas e recursos era realizada comparando-se com o sistema legado, garantindo alinhamento com os modos de operação esperados. Esse contato próximo com o \textit{stakeholder} permitiu identificar problemas rapidamente e esclarecer dúvidas relacionadas às regras de negócio inerentes à construção civil e à gestão administrativa.
