\chapter{Estado da Arte}\label{cap:art}

As técnicas de análise de dados aplicadas nas empresas são classificadas em cinco tipos: Descritiva, Diagnóstica, Preditiva, Prescritiva e Cognitiva. A classificação Descritiva explica o ocorrido, através de tecnologias de \gls{BI}. Já a Diagnóstica busca explicar a causa raiz, enquanto a Preditiva procura antecipar eventos futuros. Em sequência, a Prescritiva define a intervenção a ser feita. Por fim, a Cognitiva visa extrair conexões entre os dados \cite{Kambhampaty2021}. A integração dessas análises permite que as empresas antecipem tendências futuras, identifiquem riscos potenciais e tomem decisões baseadas em dados \cite{Agu2024}.

O setor da construção civil produz grandes volumes de dados heterogêneos e dinâmicos, fragmentados ao longo do ciclo de vida de uma obra \cite{Rodrigues2022}. No entanto, a indústria da construção civil é caracterizada pela lentidão na adoção de novas tecnologias em comparação com outros setores \cite{Lopes2020}, bem como por uma barreira cultural potencializada pela falta de mão de obra qualificada \cite{Aykanat2025}. Somado a isso, as fontes de dados incluem \textit{datasets} de simulação, experimentais, organizacionais, de monitorização de projetos e fontes não estruturadas, como redes sociais e relatórios \cite{Li2023}. A adoção de \gls{BI} no setor é aplicada para diversos fins, como estimativa de custos e orçamentos, previsão de atrasos, previsão de consumo de recursos e gestão da produtividade \cite{Ngo2020}.

Os principais problemas que persistem nos projetos de construção civil incluem a falta de abordagens unificadas de gerenciamento de dados, ineficiências no planejamento, coordenação e comunicação \cite{Wattuhewa2023}. Adicionalmente, erros de projeto levam a aproximadamente 33\% do desperdício de materiais e recursos \cite{Ram2019}. Por outro lado, isso abre espaço para aplicações de outras tecnologias, como sensores \gls{IoT}, realidade virtual, realidade aumentada, inteligência artificial, aprendizado de máquina e impressão 3D \cite{Kumar2023}. 

No escopo de \textit{Big Data}, existem três conceitos-chave: velocidade, volume e variedade \cite{Ochuba2024}. Para lidar com isso, o foco central é o \gls{BIA}. Este termo abrange metodologias utilizadas para coletar, analisar e disseminar dados, de forma a favorecer tomadas de decisão estratégicas e fundamentadas \cite{Bany2022}. Estão incluídas ferramentas \gls{ETL}, \gls{DW}, tecnologia \gls{OLAP}, \gls{DM} e relatórios textuais ou gráficos \cite{AlKhatib2021} \cite{AlOkaily2022}. Em outras palavras, cada uma delas cobre objetivos específicos. O primeiro é o \gls{ETL}, que representa a coleta, transformação e carregamento dos dados no sistema \cite{Bany2022}. Cabe ressaltar que digitalizar documentos faz parte deste grupo. Outro pilar é o \gls{DW}, repositório que armazena o grande volume de dados \cite{Daissaoui2020} e é otimizado para consultas ao acervo \cite{AlOkaily2022}. Já no contexto de \gls{DM}, esta técnica busca classificar, agrupar, associar e prever dados \cite{Ericsson2022}. Por outra frente, a abordagem \gls{OLAP} permite visualizações rápidas de dados por diversos ângulos e navegação histórica ou pelos relacionamentos entre eles \cite{Kambhampaty2021}. Já os relatórios, geralmente construídos como painés, apresentam em uma única tela os pontos mais importantes para a tomada de decisão \cite{Rodrigues2022}.

Em razão das características únicas dos projetos de construção civil, grande parte dos \textit{softwares} exige personalização, seja pela possibilidade de customização de alternativas existentes, seja pelo desenvolvimento sob demanda de aplicações específicas. No mercado português e mundial existem diversas soluções orientadas à gestão de recursos na construção civil (\gls{PRE}). Cada uma dessas ferramentas apresenta diferentes graus de maturidade, funcionalidades e custos, sendo por vezes mais ajustadas a grandes empresas do que a \gls{PME}. A Tabela \ref{tab:comparacao_softwares} apresenta uma comparação entre algumas das soluções mais relevantes, destacando vantagens e limitações. Esta análise permite contextualizar este estudo e reforça a necessidade de uma solução adaptada à realidade local da \gls{ALN}.

\begin{table}[H]
    \centering
    \caption{Comparação de soluções de software de PRE no setor da construção}
    \label{tab:comparacao_softwares}
    \begin{tabular}{|p{3cm}|p{5cm}|p{5cm}|}
    \hline
\textbf{Ferramenta} & \textbf{Vantagens} & \textbf{Limitações} \\ \hline

Primavera Construction & 
Integra contabilidade, gestão de obras e RH; forte presença em Portugal; adequado para PME. & 
Licenciamento relativamente caro; curva de aprendizagem para utilizadores menos experientes. \\ \hline

PHC CS Construção & 
Gestão de contratos, empreitadas e faturação; interface amigável; solução nacional. & 
Menos flexível para integração com plataformas externas; custo médio. \\ \hline

ETICADATA Obras & 
Boa adaptação à realidade portuguesa; módulos de orçamento e acompanhamento de obra. & 
Menor suporte para análises avançadas e integração com \gls{BI}. \\ \hline

SAP Business One (Construção) & 
Altamente robusto; integra finanças, compras, logística e construção; escalável para grandes empresas. & 
Custos muito elevados; elevado tempo de implementação; pouco ajustado a PME. \\ \hline

Microsoft Dynamics 365 (+ Add-ons) & 
Flexível; integração nativa com Power BI; forte capacidade analítica e modular. & 
Licenciamento elevado; necessidade de customização para o setor da construção. \\ \hline

Autodesk Construction Cloud & 
Colaboração em tempo real; suporte BIM; integração de documentação e cronogramas. & 
Mais orientado para gestão de projetos do que para gestão financeira; custos de subscrição. \\ \hline

Procore & 
Foco em gestão de projetos; comunicação eficiente entre equipas; solução \textit{cloud}. & 
Mais popular em mercados internacionais; custo elevado; pouca adaptação ao contexto português. \\ \hline

Buildertrend & 
Orientado a PME; interface intuitiva; acesso em nuvem; foco em orçamentos e cronogramas. & 
Funcionalidades financeiras limitadas; menor presença na Europa; dependência da \textit{cloud}. \\ \hline

\end{tabular}
\end{table}
