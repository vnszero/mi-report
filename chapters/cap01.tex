\chapter{Introdução}\label{cap:intro}

No contexto de empresas de construção civil no distrito de Bragança em Portugal, existe a empresa \gls{ALN}, uma \gls{PME} da região especializada em execução de obras públicas e privadas. Nessa esfera, gerir recursos e processos de forma eficiente é fundamental para garantir competitividade nesse ramo de negócio. Para isso, existem \textit{softwares} com foco em \gls{PRE}.

Entre 2023 e 2024, uma nova parceria entre a \gls{ALN} e o \gls{IPB} resultou no desenvolvimento de uma nova plataforma web com foco em \gls{PRE}. Esta nova solução foi baseada em um projeto de 2013, entretanto, a nova versão não foi oficialmente para produção por não possuir boa parte das funcionalidades essenciais que a antiga ferramenta era capaz de cobrir. Frente a necessidade de substituir a plataforma obsoleta por tecnologias modernas e complementar o produto surge uma terceira parceria entre os anos de 2024 e 2025.

O objetivo geral desta nova etapa está em corrigir, completar funcionalidades ausentes e integrar novos módulos de \textit{business analytics} de modo a suportar uma gestão inteligente e sustentável da construção. Já os objetivos específicos incluem desenvolvimento do Módulo de Gestão de Obras, do Módulo de Parametrização, do Módulo de Registo de Ponto e do Módulo de Gestão Inteligente e Sustentável da Construção.

Entre as ferramentas neste contexto de implementação estão Primavera Contruction, PHC CS Construção e Eticadata Obras. Todas oferecem funcionalidades de gestão orçamental, contabilística e acompanhamento de projetos. Existem também alternativas mais robustas como SAP Business One, Microsoft Dynamics 365, Autodesk Construction Cloud, Procore e Buildertrend que para além dos recursos básicos trazem colaboração em tempo real, metodologias \gls{BIM} e integração com análises preditivas. Embora estas soluções apresentem maturidade, elas implicam custos elevados de licenciamento, necessidade de adaptação à realidade local ou ausência de módulos específicos, o que justifica o desenvolvimento de uma plataforma ajustada à realidade da \gls{ALN}.

Os próximos capítulos desta dissertação seguem modelo estruturado. O Estado da Arte será apresentado no Capítulo \ref{cap:conceptual} e será feita comparações com ferramentas pré-disponíveis no mercado nesse setor. Já no Capítulo \ref{cap:metodology}, serão explorados os detalhes sobre as correções necessárias, os módulos que faltavam e as histórias de usuário dos novos recursos. Em seguida, no Capítulo \ref{cap:development}, são apresentados o desenvolvimento do projeto, as mudanças que foram necessárias no banco de dados, alterações do frontend e backend. No Capítulo \ref{cap:test}, são detalhados os resultados obtidos neste projeto. Por fim, no Capítulo \ref{cap:conclusions}, estão as conclusões do presente trabalho.