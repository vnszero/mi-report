\chapter{Introdução}\label{cap:intro}

O Capítulo \ref{cap:intro} é dedicado a uma introdução ao tema do trabalho, descrevendo as ideias gerais do problema em foco e a sua importância. Devem ainda ser explicitados os objetivos do trabalho, clarificada a estrutura do relatório e indicadas as convenções tipográficas.

\section{Enquadramento}

Deve haver um enquadramento introdutório, que descreva o contexto em que o trabalho se insere, referenciando a proposta original do projeto, que deve constar no primeiro apêndice do documento (ver apêndice \ref{apendice1}).

\section{Objetivos}

Os objetivos do trabalho devem ser apresentados de forma clara e compatível com a proposta original do projeto. Na eventualidade de os objetivos originais terem sido reformulados, devem ser apresentadas as razões objetivas que conduziram a essa reformulação.

Idealmente, deve-se incluir um cronograma do projeto, indicando explicitamente as tarefas realizadas e o tempo dedicado a cada uma. Existindo um cronograma na proposta original do projeto, deverão justificar-se eventuais discrepâncias com o cronograma real.


\section{Estrutura do Documento}

Este modelo de relatório assume que a maioria dos projetos de fim de curso são centrados no desenvolvimento de uma solução informática para um problema. Sendo esse contexto, é apropriada uma estrutura que descreva a análise, conceção e desenvolvimento da solução implementada. Em particular, em projetos de desenvolvimento de software, espera-se que o relatório documente as principais fases do ciclo de desenvolvimento.

Nos casos em que o trabalho corresponde sobretudo à integração e/ou avaliação de componentes pré-existentes, a estrutura do relatório deverá ser adaptada em conformidade, com ênfase na descrição das tecnologias subjacentes, sua articulação e avaliação.

A estrutura efetivamente adotada para o resto do relatório é, normalmente, clarificada nesta secção, usando texto semelhante a: ``O resto do relatório está organizado da seguinte forma: no capítulo 2 descreve-se ...; no capítulo 3 ...; ...; finalmente, o último capítulo  apresenta as conclusões e direções de trabalho futuro.''

\section{Normas de Composição}

Para além de uma organização que reflita o percurso seguido, o relatório deve estar bem formatado e ter aspeto sóbrio, convidando à leitura e fazendo jus ao mérito do trabalho descrito. Neste sentido, apresentam-se de seguida algumas das normas a levar em conta\footnote{Estas normas, assim como este modelo de documento, são compatíveis com o preconizado no "Regulamento da Unidade Curricular de Projecto das Licenciaturas", designadamente no que diz respeito ao relatório do projeto.}.

\subsubsection{Convenções Tipográficas}

Por vezes, opta-se por apresentar as convenções tipográficas seguidas no documento, ou seja, em que circunstâncias se usam texto em \textit{itálico}, \textbf{negrito}, ou de \texttt{espaçamento uniforme} (esta última formatação é normalmente usada para apresentar código fonte), bem como quais as fontes tipográficas usadas, respetivas dimensões, etc.


\subsubsection{Tabelas e Figuras}

Tabelas e figuras devem ser numeradas automaticamente e ter um tamanho equilibrado (nem muito grande, nem muito pequeno), como a Tabela \ref{tab:exemplo_de_tabela} e as Figuras  \ref{fig:exemplo_de_figura}, \ref{fig:exemplo_de_figura2} e \ref{fig:exemplo_de_figura3}.

\begin{table}[htbp]	
	\centering
	{\small
		\begin{tabulary}{\linewidth}{|L|C|R|}	
			\hline 	
			{\bf Nome da Coluna 1} & {\bf Nome da Coluna 2} & {\bf Nome da Coluna 3} \\ 
			\hline 
			conteúdo A & conteúdo B & conteúdo C \\ 
			\hline 
			conteúdo D & conteúdo E & conteúdo F \\ 
			\hline 
		\end{tabulary} 
	}	
	\caption{\small{Exemplo de tabela.}}
	\label{tab:exemplo_de_tabela}
\end{table}


\begin{figure}[htbp]
	\centering
	\includegraphics[scale=0.75]{images/lion_large}
	\caption{Exemplo de imagem PNG.}
	\label{fig:exemplo_de_figura}
\end{figure}

Sempre que possível, devem-se usar formatos escaláveis (e.g., PDF), como na Figura \ref{fig:exemplo_de_figura2}, e evitar imagens comprimidas (formatos JPG, PNG, GIF, etc.), como na Figura \ref{fig:exemplo_de_figura}.

\begin{figure}[htbp]
	\centering
	\includegraphics[width=0.75\linewidth]{images/architecture}
	\caption{Exemplo de imagem PDF.}
	\label{fig:exemplo_de_figura2}
\end{figure}


Em gráficos devem-se indicar sempre as grandezas associadas a cada eixo, bem como a respetiva legenda -- ver exemplo na Figura \ref{fig:exemplo_de_figura3}. Adicionalmente, o esquema de cores ou de traços para as linhas, deve ser sóbrio e prevenir ambiguidades na leitura do gráfico.

\begin{figure}[htbp]
	\centering
	\includegraphics[width=0.65\linewidth]{images/search_times}
	\caption{Exemplo de gráfico.}
	\label{fig:exemplo_de_figura3}
\end{figure}

\subsubsection{Distribuição dos Elementos}
A forma como o texto e outros elementos (tabelas, figuras, etc.) se distribui por uma página, deve ser tal que se evitem grandes blocos vazios no final da página. Embora algumas sistemas de composição tendam a garantir isso automaticamente (como o LaTeX), é costume serem necessárias afinações para resolver, manualmente, essas (e outras) desformatações. No entanto, essas afinações devem ser deixadas para a fase pré-impressão, já com o conteúdo do documento estabilizado, de forma a evitar trabalho inconsequente.


\subsubsection{Correcção Ortográfica}
Para além da qualidade tipográfica do relatório, é imprescindível minimizar (se possível até, erradicar) erros ortográficos. Qualquer sistema de composição de documentos suporta correção ortográfica (e, muitas vezes, sintática), pelo que não é aceitável a submissão para avaliação de relatórios sem revisão ortográfica prévia.


\subsubsection{Referências Bibliográficas}
Ao longo do documento, deve ficar sempre perfeitamente claro o que é escrita original e o que foi baseado (ou até reproduzido de) noutras fontes. 

Todas as fontes devem ser descritas na formatação usada na Bibliografia e referenciadas, no texto, pelos seus identificadores únicos. Por exemplo: ``Uma solução para o problema em causa deve respeitar as propriedades {\it x}, {\it y} e {\it z} \cite{INPROC1}.'', ou ``Neste trabalho explorou-se a API PThreads \cite{TECH01} com o objetivo de $\dots$''.  No modelo em LaTeX, as referências bibliográficas são definidas no ficheiro {\tt libs/refs.bib}, onde existem entradas de diferentes tipos: livro \cite{BOOK01}, artigo de conferência \cite{INPROC1}, relatório técnico \cite{TECH01} e sítio web \cite{FORD11}.

A reprodução fiel de texto de fontes externas deve ser limitada, surgir entre aspas, ligeiramente destacada, e ter apensa a respetiva referência bibliográfica.  Por exemplo:

\begin{quotation}
``Recently, the employment of GPU devices is a key to achieve higher performance for computer systems. On those systems, GPUs are used for general calculation but with extreme parallelism.'' \cite{INPROC1}
\end{quotation}

\subsubsection{Siglas}
Na primeira vez as siglas devem surgir por extenso, sendo resumidas nas vezes seguintes. Por exemplo: ``o curso atual de Engenharia Informática da \gls{ESTiG} foi reformulado em 2015 e, a par com o curso de Informática de Gestão, representa o leque de licenciaturas da área de Informática que a \gls{ESTiG} oferece''. No modelo em LaTeX, as siglas são definidas no ficheiro {\tt acronym.tex}.



