\chapter{Introdução}\label{cap:intro}

No contexto de empresas de construção civil no distrito de Bragança, em Portugal, existe a empresa \gls{ALN}, uma \gls{PME} da região especializada na execução de obras públicas e privadas. Nesse cenário, gerir recursos e processos de forma eficiente é fundamental para garantir competitividade nesse ramo de negócio. Para isso, existem \textit{softwares} com foco em \gls{PRE}.

Entre 2023 e 2024, uma nova parceria entre a \gls{ALN} e o \gls{IPB} resultou no desenvolvimento de uma nova plataforma web com foco em \gls{PRE}. Esta solução foi baseada em um projeto de 2013. Entretanto, a nova versão não foi oficialmente disponibilizada em produção por não possuir boa parte das funcionalidades essenciais que a antiga ferramenta era capaz de cobrir. Frente à necessidade de substituir a plataforma obsoleta por tecnologias modernas e complementar o produto, surgiu uma terceira parceria entre os anos de 2024 e 2025.

O objetivo geral desta nova etapa consiste em corrigir e completar funcionalidades ausentes, além de integrar novos módulos de \gls{BIA}, de modo a suportar uma gestão inteligente e sustentável da construção. Já os objetivos específicos incluem o desenvolvimento do Módulo de Gestão de Obras, do Módulo de Parametrização, do Módulo de Registo de Ponto e do Módulo de Gestão Inteligente e Sustentável da Construção.

Entre as ferramentas disponíveis nesse contexto de implementação estão Primavera Construction, PHC CS Construção e Eticadata Obras. Todas oferecem funcionalidades de gestão orçamental, contabilística e acompanhamento de projetos. Existem também alternativas mais robustas, como SAP Business One, Microsoft Dynamics 365, Autodesk Construction Cloud, Procore e Buildertrend, que, para além dos recursos básicos, oferecem colaboração em tempo real, metodologias \gls{BIM} e integração com análises preditivas. Embora estas soluções apresentem maturidade, elas implicam custos elevados de licenciamento, necessidade de adaptação à realidade local ou ausência de módulos específicos, o que justifica o desenvolvimento de uma plataforma ajustada à realidade da \gls{ALN}.

Os próximos capítulos desta dissertação seguem um modelo estruturado. O Estado da Arte será apresentado no Capítulo \ref{cap:art}, onde serão feitas comparações com ferramentas disponíveis no mercado nesse setor. Já no Capítulo \ref{cap:metodology}, serão explorados os detalhes sobre as correções necessárias, os módulos em falta e as histórias de usuário dos novos recursos. Em seguida, no Capítulo \ref{cap:development}, são apresentados o desenvolvimento do projeto, as mudanças necessárias no banco de dados e as alterações no frontend e backend. No Capítulo \ref{cap:test}, são detalhados os resultados obtidos neste projeto. Por fim, no Capítulo \ref{cap:conclusions}, são apresentadas as conclusões do presente trabalho.
