\chapter{Implementação}\label{cap:development}

As 85 tarefas mencionadas anteriormente distribuíram-se em 78 tarefas destinadas a melhorias, a correções, a refatoração de código e a disponibilização de páginas já modeladas, porém ainda indisponíveis, e 7 tarefas relacionadas a novas funcionalidades de \gls{BI}.

\section{Contexto}

Os primeiros desafios envolveram compreender o \textit{setup} inicial da aplicação e fazê-la funcionar, uma vez que havia pouca documentação sobre a configuração adotada pelo antigo desenvolvedor. Paradoxalmente, essa etapa revelou-se uma das mais complexas do projeto, sendo plenamente resolvida apenas após contato direto com o ex-aluno responsável pelo desenvolvimento inicial.

Superada essa fase, procedeu-se à identificação das páginas ainda inexistentes, o que permitiu a criação de tarefas específicas para cobrir todas as lacunas funcionais. Ao todo, faltavam 17 ferramentas. No grupo Equipamentos \& Serviços, faltavam 3: \textbf{Equipamento}, \textbf{Viatura} e \textbf{Tipo}. No contexto de Obras, faltavam 2: \textbf{Nota de Encomenda} e \textbf{Nota de Crédito Material}. Em Outros Custos havia 1 pendente: \textbf{Seguro de Responsabilidade Civil}. Em Recursos Humanos, faltavam 3: \textbf{Descendentes de Colaboradores}, \textbf{Salários} e \textbf{Cálculo de Vencimento}.

Além disso, existiam grupos inteiros sem qualquer implementação prévia: \textbf{Amortizações Fiscais}, \textbf{Entidades} e \textbf{Parametrização}. Em Amortizações Fiscais deveriam ser desenvolvidas 3 ferramentas: \textbf{Amortização de Viatura}, \textbf{Amortização de Equipamento} e \textbf{IVA Devolvido/Crédito}. Em Entidades, eram necessárias outras 3: \textbf{ALN}, \textbf{Fornecedores} e \textbf{Clientes}. Por fim, na área de Parametrização cabia implementar as 2 ferramentas \textbf{Parâmetros} e \textbf{Valores dos Parâmetros}.

\section{Tarefas Antes dos Testes em Campo}

As etapas seguintes incluíram a correção do gerador de PDF para a folha de presenças. A biblioteca anteriormente utilizada, \textit{QuestPDF}, apresentava falhas, motivo pelo qual foi substituída pela \textit{iText7}. Como as bibliotecas não eram compatíveis, todo o relatório precisou ser reescrito, o que exigiu refatoração completa do mecanismo de geração de documentos.

Ainda na página de presenças, o mecanismo de replicação apresentava comportamento inadequado em razão do tratamento de datas em \textit{JavaScript}. Para eliminar inconsistências, adotou-se a biblioteca \textit{DayJS}. Além disso, devido à modelagem original da base de dados e às regras de negócio, o sistema de registro de presenças necessitou revisão. O modelo antigo distinguia presença e falta com valores numéricos e a letra “F”, mas não representava presenças parciais. O novo modelo passou a utilizar: \textbf{n} para presença regular, \textbf{0} para falta e \textbf{-n} para presença parcial, uma vez que o número n identifica a obra. Esse modo de operação representa uma regra de negócio já consolidada e não devia ser alterada.

Outra correção importante envolveu os cálculos da ferramenta \textbf{Fatura Obra}, cujos subtotais eram incorretos por descumprimento da ordem adequada das operações ao considerar impostos e descontos. Também se identificou que o padrão de usabilidade dos formulários incluía uma página extra de confirmação de envio, gerando cliques desnecessários. Assim, foram criadas tarefas de ajuste para 17 das 18 ferramentas já existentes. Já na página de Inspeção Periódica, verificou-se que datas estavam registradas em formatos divergentes (por exemplo, \texttt{18/07/2011} em vez de \texttt{18-07-2011}). O sistema legado tratava datas como texto, mas o novo sistema utiliza tipos de dados de data, exigindo ajustes nos dados migrados.

As quatro primeiras páginas totalmente novas desenvolvidas foram: \textbf{ALN}, \textbf{Fornecedores}, \textbf{Clientes} e \textbf{Cálculo de Vencimento}. Essa ordem permitiu crescimento progressivo de complexidade, com as três primeiras servindo como preparação técnica para a última. Durante esses desenvolvimentos, identificou-se que a barra de pesquisa sobrepunha botões na interface, motivando ajustes no estilo.

As demais 13 tarefas foram dedicadas à implementação das ferramentas restantes: \textbf{Equipamento}, \textbf{Viatura}, \textbf{Tipo}, \textbf{Nota de Encomenda}, \textbf{Nota de Crédito Material}, \textbf{Seguro Responsabilidade Civil}, \textbf{Amortização Viatura}, \textbf{Amortização Equipamento}, \textbf{IVA Devolvido/Crédito}, \textbf{Descendentes Colaboradores}, \textbf{Salários}, \textbf{Parâmetros} e \textbf{Valores dos Parâmetros}.

Por fim, havia um problema de infraestrutura relativo ao servidor de produção, executado em \textit{Windows 11 Home Edition}. Após reinicializações, o Docker não era iniciado automaticamente, deixando o sistema indisponível até que alguém efetuasse \textit{login}. Como solução, utilizou-se o \textit{Task Scheduler} do Windows para realizar \textit{login} automático, iniciar o Docker e, em seguida, bloquear a sessão. Embora não resolva reinicializações motivadas por atualizações automáticas, o procedimento cobre casos como quedas de energia.

\section{Tarefas em Paralelo com Testes Junto ao Cliente}

Com as correções preliminares concluídas, o sistema pôde ser disponibilizado em produção, permitindo o surgimento natural de ajustes decorrentes do uso real. Um dos primeiros pontos corrigidos foi o valor padrão do imposto, que precisava ser pré-carregado nos formulários de todas as ferramentas. Problemas nos mecanismos de busca, tanto internas quanto externas aos formulários, também demandaram melhorias para garantir boa experiência de uso. Ajustes nos documentos gerados eram necessários para assegurar que ocupassem preferencialmente uma única página.

Campos monetários passaram a exigir suporte a valores com até 4 casas decimais, mas escolhas anteriores de componentes no \textit{frontend} impediam esse comportamento. A solução adotada foi a remoção das máscaras de preenchimento. Campos relacionados a acertos também foram reposicionados das telas de formulário para as listagens, onde possuíam maior relevância. Outro campo ausente era o de desconto geral da fatura, uma vez que esse tipo de desconto é previsto nas regras de negócio. Houve também pedidos para redução de tamanhos de fonte e compressão de colunas, otimizando o espaço de visualização e reduzindo a necessidade de rolagem horizontal.

No âmbito da busca, além das pesquisas textuais, tornou-se necessária a inclusão de filtros por intervalo de datas, desconsideração de caracteres especiais e ampliação dos critérios de pesquisa. Uma validação adicional foi implementada no número da fatura para evitar duplicidades entre fornecedores distintos. No filtro de obras, o mecanismo de determinação de obras ativas precisou ser corrigido: embora houvesse um campo de \textit{status}, o sistema também considerava a data de término, o que gerava falsos negativos em obras reativadas por motivos de garantia.

Na ferramenta de Nota de Crédito Material, o mecanismo de seleção de materiais estava defeituoso devido à ausência de parâmetros no \gls{DTO} utilizado, exigindo correção. O campo de quantidade também precisou passar a aceitar valores decimais, conforme previsto no banco de dados. Na página de presenças, adicionou-se um informativo sobre os modos de uso. Já na página de Cálculo de Vencimento, corrigiram-se problemas na paginação, adequou-se o suporte a valores negativos nos campos de ajuste e corrigiu-se o cálculo exibido no relatório impresso. Problemas adicionais surgiram na página de Execução Orçamental. É importante salientar que, embora a maioria dos ajustes tenha ocorrido no \textit{frontend}, muitas das alterações impactaram também o \textit{backend}.

\section{Novos Módulos de Business Intelligence}

Em reunião com o \textit{stakeholder}, foram definidos os novos indicadores de desempenho que integrariam o módulo de \gls{BI}. Em âmbito global, foram estabelecidos quatro indicadores: obras ativas, obras finalizadas, cliente com mais obras, número de obras do cliente com mais obras, obras com maiores custos de materiais, classificação de materiais, classificação de clientes e classificação de fornecedores, portanto esses dados juntos integrariam a ferramenta Indicadores Globais. Em âmbito específico de obra, foram definidos dois indicadores: tempo de extrapolação do previsto e comparação entre orçamento e custo real registrado. Já para o indicador de extrapolação temporal, ficou estabelecido que as obras deveriam ser divididas em capítulos, cada qual com orçamento próprio e datas previstas e reais.
