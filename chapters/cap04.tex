\chapter{Implementação}\label{cap:development}

As 85 tarefas já mencionadas estão distribuídas em 78 tarefas de melhorias, correções, refatoração de código e páginas que já estavam modeladas, mas não estavam disponíveis e 7 tarefas relacionadas com novas funcionalidades de \gls{BI}. 

\section{Contexto}

Os primeiros dois desafios foram descobrir como era o \textit{setup} inicial da aplicação e como fazê-la funcionar, pois havia pouca documentação disponível de como era a configuração do antigo desenvolvedor. Por mais irônico que pareça, essa foi uma das tarefas mais difíceis de todo projeto e só se resolveu por definitivo após entrar em contato direto com o ex-aluno. O próximo passo consistiu em identificar as páginas que estavam faltando e criar tarefas para cobrir todas elas. Ao todo, faltavam 17 ferramentas. No grupo de Equipamentos & Serviços faltavam 3: Equipamento, Viatura e Tipo. Já no contexto de Obras, faltavam 2: Nota Encomenda e Nota Crédito Material. Na próxima coleção Outros Custos, faltava 1: Seguro Responsabilidade Civil. Na coleção de Recursos Humanos faltavam 3: Descendentes Colaboradores, Salários e Cálculo Vencimento. Além disso haviam grupos que não tinham nenhuma ferramenta pronta: Amortizações Fiscais, Entidades e Parametrização. Em Amortizações Fiscais deveriam ser implementadas 3: Amortização Viatura, Amortização Equipamento e IVA Devolvido/Credito. Já em Entidades, era necessário implementar 3: ALN, Fornecedores e Clientes. Por fim, cabia desenvolver 2 em Parametrização: Parâmetros e Valores dos Parâmetros.

\section{Tarefas Antes dos Testes em Campo}

Os próximos passos eram corrigir o PDF gerado para a folha de presenças. Aconteceu que a antiga biblioteca \textit{QuestPDF} usada para gerar esses arquivos não funcionava e, portanto, ela foi substituida pela \textit{iText7}. A parte ruim de fazer essa substituição é que todo o arquivo que gerava o relatório precisou ser refeito por completo, uma vez que as bibliotecas não eram compatíveis entre si e funcionavam de formas diferentes. Por outra frente, ainda na página de presenças, o mecanismo que replicava presenças apresentava um comportamento inadequado em consequência da forma como datas eram manipuladas no \textit{JavaScript}, a biblioteca \textit{DayJS} foi adotada para evitar comportamentos inesperados. Ainda sobre esta página, pela forma como foi pensado o banco de dados e regra de negócio, o modo de operação das presenças precisou ser modificado, pois não cobria todos os casos de uso do cliente. O modelo antigo sugeria usar um número para identificar presença na obra de mesmo número e a letra 'F' para faltas. Entretanto, isso não identificava as presenças parciais. Portanto, o sistema de presenças passou a ser número 'n' para presença regular, '0' para faltas e '-n' para presença parcial em determinada obra. 

Outra tarefa foi corrigir os cálculos dentro da ferramenta de Fatura Obra que apontava montantes errados para os subtotais apresentados ao considerar impostos ou descontos, a ordem das operações não era respeitada. Somado a isso, havia um problema na escolha de projeto para conclusão dos formulários presentes nas ferramentas. Todo formulário possuia uma página extra de confirmação de envio das informações preenchidas. Isso representava um clique extra em todos os preenchimentos de formulário. Das 17 das 18 ferramentas já existentes, foi necessário abrir uma tarefa de ajuste do modal para cada. Já na página de Inspeção Periódica, parte dos campos de datas estava salva fora do padrão esperado, 18/07/2011 ao invés de 18-07-2011 por exemplo. O \textit{software} de 2013 tratava datas como texto e suportava essas variações, mas o novo sistema foi projetado para trabalhar com variáveis tipo data e isso precisou ser ajudatado na migração de dados do antigo \textit{backup}.

As 4 primeiras novas páginas a serem programadas foram ALN, Fornecedores, Clientes e Cálculo Vencimento, nessa ordem. Essa sequência representava um crescimento gradual de dificuldade de implementação e as três primeiras funcionaram como um treinamento para a última. Com as novas telas, foi identificado que a barra de pesquisa para a folha de estilo adotada entrava em sobreposição com botões, por isso foi adequado tratar esse problema. As próximas 13 tarefas eram dedicadas a implementar as outras 13 páginas que faltavam: Equipamento, Viatura, Tipo, Nota Encomenda, Nota Crédito Material, Seguro Responsabilidade Civil, Amortização Viatura, Amortização Equipamento, IVA Devolvido/Credito, Descendentes Colaboradores, Salários, Parâmetros e Valores dos Parâmetros

Havia também um problema de Infraestrutura relacionado ao servidor de produção. A máquina disponível é um Windows 11 Home Edition, ou seja, em caso de reinicialização forçada do servidor, o Docker não reiniciava e o sistema inteiro ficava indisponível até que alguém fizesse o \textit{login}. Isso é um problema comum uma vez que esse Sistema Operacional não é feito para esse fim. A solução foi utilizar o recurso de \textit{tasks} do Windows. Toda vez que a máquina liga, ela faz um login automático, inicia o Docker e, após isso, dispara um comando que bloqueia a tela. Não é uma solução perfeita, pois não é capaz de lidar com atualizações automáticas do Windows, mas consegue cobrir outros casos como reinicialização por queda de energia.

\section{Tarefas em Paralelo com Testes Junto ao Cliente}

Com todas as correções preliminares feitas, o sistema pôde ser lançado para produção e naturalmente surgiriam problemas e detalhes a serem corrigidos. O primeiro ponto era com relação ao valor padrão da porcentagem referente ao imposto que devia ser pré-carregado nos formulários e não estava para todas as ferramentas. Além disso, haviam problemas nos mecanismos de busca dentro e fora dos formulários que não permitiam boa experiência de usuário. Ajustes também foram necessários nos docuementos gerados porque era conveniente que eles ocupassem apenas 1 página para impressão.

Em outra frente, campos monetários deviam suportar valore com até 4 casas decimais, entretanto escolhas de projeto para o \textit{frontend} não davam suporte adequado para isso. Portanto, a melhor solução foi a remoção definitiva das máscaras de preenchimento. Já os campos relacionados com acertos apresentavam mais sentido junto as listagens e não nos formulários, logo foram movidos. Campos como desconto geral da fatura também estavam faltando, porque essa era uma das possíveis formas de desconto oferecidas nas regras de negócio e não um item calculado sobre os descontos acumulados de todos os materiais. Tarefas relacionadas com redução dos tamanhos de fonte e compressão de colunas também foram solicitadas para otimizar o espaço disponível dos monitores e evitar rolagem lateral.

Ainda sobre mecanismos de busca, para além das pesquisas textuais, foi solicitada busca por intervalo de datas, desconsideração de caracteres especiais e inclusão de mais campos nos cretérios de busca. Outra importante alternativa era a validação do número da fatura em concordância com o fornecedor e data de preenchimento. Isso para evitar dados repetidos, mas sem criar conflitos entre diferentes fornecedores. Já no filtro de obras, era necessário mudar o mecanismo que considerava obras ativas ou não, estava previsto campo de \textit{status}, entretanto, o campo de término da obra, também era levado em conta, porém isso seria errado, há casos de obras já encerradas que por motivos de garantia, precisam ser reativadas.

Foi identificado mal funcionamento no mecanismo de seleção dos nomes dos materiais no formulário de Nota de Crédito, os itens estavam sem nome devido a falta do parâmetro na estrutura do \gls{DTO} utilizado por essa ferramenta e precisou ser corrigido. Outro detalhe é que o campo de quantidade precisava suportar valores decimais, isso era previsto pelo banco de dados, mas o havia estruturas impeditivas nos formulários. Já na página de presenças entendeu-se importante adicionar um informativo sobre os modos de uso da presença.

No perspectiva da página de Cálculo Vencimento foi feita ajuste no sistema de paginação, adequação dos campos de ajuste para que dessem suporte à valores negativos e o cálculo da coluna de vencimentos estava errada no relatório impresso. Problemas com valores encontrados também surgiram na página de Execução Orçamental. Cabe ressaltar que os problemas que apareceram até o momento provocavam mudanças majoritariamente no \textit{frontend}, mas podiam repercutir mudanças até no \textit{backend}.

\section{Novos Módulos de Business Intelligence}

Em reunião com o \textit{stakeholder} foi possível definir quais seriam os novos indicadores de desempenho a serem integradas no aspecto de \gls{BI}. Em aspecto global são 4 ao todo: obras ativas por ano, obras com maior consumo de materiais, \textit{ranking} de materiais, \textit{ranking} de fornecedores. Já em caráter de obras específicas, são 2 indicadores: tempo de extrapolação previsto e comparação entre o orçamento em relação ao custo real registrado. No caso do indicador de extrapolação temporal, ficou definido que seria adequado dividir as obras por capítulos, cada capítulo com seu orçamento e datas previstas e reais.