\chapter{Discussão}\label{cap:test}

Após as implementações citadas no capítulo anterior, os grupos pré-definidos Equipamentos e Serviços e Obras ficaram como é possível ver nas Figuras \ref{equipamentos-e-servicos} e \ref{obras}. 

\begin{figure}[H]
    \centering 
    \includegraphics[width=0.4\textwidth]{images/equipamentos-e-servicos.png} 
    \caption{Equipamentos & Serviços - Final} 
    \label{equipamentos-e-servicos}
\end{figure}

\begin{figure}[H]
    \centering 
    \includegraphics[width=0.4\textwidth]{images/obras.png} 
    \caption{Obras - Final} 
    \label{obras}
\end{figure}

O mesmo se aplica para as agrupamentos Outros Custos e Recursos Humanos, como pode ser visto nas Figuras \ref{outros-custos} e \ref{recursos-humanos}. Cabe ressaltar que a página Serviços e Consumíveis foi renomeada para Fatura Serviços e Consumíveis para melhorar a distinção nominal em relação a página do grupo Equipamentos e Serviços.

\begin{figure}[H]
    \centering 
    \includegraphics[width=0.4\textwidth]{images/outros-custos.png} 
    \caption{Outros Custos - Final} 
    \label{outros-custos}
\end{figure}

\begin{figure}[H]
    \centering 
    \includegraphics[width=0.4\textwidth]{images/recursos-humanos.png} 
    \caption{Recursos Humanos - Final} 
    \label{recursos-humanos}
\end{figure}

Os novos grupos Amortizações Fiscais, Entidades e Parametrizações já mencionados ficaram como é possível visualizar nas Figuras \ref{amortizacoes-fiscais}, \ref{entidades} e \ref{parametrizacoes}. 

\begin{figure}[H]
    \centering 
    \includegraphics[width=0.4\textwidth]{images/amortizacoes-fiscais.png} 
    \caption{Amortizações Fiscais - Final} 
    \label{amortizacoes-fiscais}
\end{figure}

\begin{figure}[H]
    \centering 
    \includegraphics[width=0.4\textwidth]{images/entidades.png} 
    \caption{Entidades - Final} 
    \label{entidades}
\end{figure}

\begin{figure}[H]
    \centering 
    \includegraphics[width=0.4\textwidth]{images/parametrizacoes.png} 
    \caption{Parametrizações - Final} 
    \label{parametrizacoes}
\end{figure}

Em relação aos testes de disponibilidade do sistema. As duas vezes que o servidor ficou fora do ar foram em razão de atualização automática do Windows 11. Esse problema não pode ser contornado pelas tarefas de iniciazação do Sistema Operacional e é necessária intervenção humana. Já o segundo caso foi em razão da hibernação automática do Windows 11, mas essa situação pode ser resolvida com ajuste das configurações.