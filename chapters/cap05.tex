\chapter{Discussão}\label{cap:test}

Após as implementações apresentadas no capítulo anterior, o número total de grupos subiu para um total de 8, ver Figura \ref{todos-os-grupos}. Os grupos pré-definidos \textbf{Equipamentos e Serviços} e \textbf{Obras} passaram a apresentar a organização demonstrada nas Figuras \ref{equipamentos-e-servicos} e \ref{obras}. Cabe destacar que o grupo de Obras passou a integrar a ferrameta Capítulos por Obra, entretanto, Execução Orçamental foi movida para o grupo de Painéis. 

\begin{figure}[H]
    \centering 
    \includegraphics[width=0.4\textwidth]{images/todos-os-grupos.png} 
    \caption{Todos os Grupos - Versão Final} 
    \label{todos-os-grupos}
\end{figure}

\begin{figure}[H]
    \centering 
    \includegraphics[width=0.4\textwidth]{images/equipamentos-e-servicos.png} 
    \caption{Equipamentos \& Serviços - Versão Final} 
    \label{equipamentos-e-servicos}
\end{figure}

\begin{figure}[H]
    \centering 
    \includegraphics[width=0.4\textwidth]{images/obras.png} 
    \caption{Obras - Versão Final} 
    \label{obras}
\end{figure}

O mesmo ocorre com os agrupamentos \textbf{Outros Custos} e \textbf{Recursos Humanos}, conforme ilustrado nas Figuras \ref{outros-custos} e \ref{recursos-humanos}. Cabe ressaltar que a página \textit{Serviços e Consumíveis} foi renomeada para \textit{Fatura Serviços e Consumíveis}, de modo a melhorar a distinção nominal em relação à página homônima presente no grupo Equipamentos e Serviços.

\begin{figure}[H]
    \centering 
    \includegraphics[width=0.4\textwidth]{images/outros-custos.png} 
    \caption{Outros Custos - Versão Final} 
    \label{outros-custos}
\end{figure}

\begin{figure}[H]
    \centering 
    \includegraphics[width=0.4\textwidth]{images/recursos-humanos.png} 
    \caption{Recursos Humanos - Versão Final} 
    \label{recursos-humanos}
\end{figure}

Os novos grupos \textbf{Amortizações Fiscais}, \textbf{Entidades} e \textbf{Parametrizações}, já mencionados no capítulo anterior, ficaram organizados como apresentado nas Figuras \ref{amortizacoes-fiscais}, \ref{entidades} e \ref{parametrizacoes}. Destaque especial para ferramenta de Capítulo que agora integra o conjunto de Parametrizações.

\begin{figure}[H]
    \centering 
    \includegraphics[width=0.4\textwidth]{images/amortizacoes-fiscais.png} 
    \caption{Amortizações Fiscais - Versão Final} 
    \label{amortizacoes-fiscais}
\end{figure}

\begin{figure}[H]
    \centering 
    \includegraphics[width=0.4\textwidth]{images/entidades.png} 
    \caption{Entidades - Versão Final} 
    \label{entidades}
\end{figure}

\begin{figure}[H]
    \centering 
    \includegraphics[width=0.4\textwidth]{images/parametrizacoes.png} 
    \caption{Parametrizações - Versão Final} 
    \label{parametrizacoes}
\end{figure}

O novo grupo de Painéis, contém os Indicadores Globais, os Indicadores Específicos e a Execução Orçamental foi movida para este grupo pois faz sentido dispor no mesmo lugar todas as ferramentas de BI. Isso pode ser observado na Figura \ref{paineis}. As versões finais das ferramentas Indicadores Globais e Indicadores Específicos podem ser vistas nas Figuras \ref{indicadores-globais} e \ref{indicadores-especificos} respectivamente.

\begin{figure}[H]
    \centering 
    \includegraphics[width=0.4\textwidth]{images/paineis.png} 
    \caption{Painéis - Versão Final} 
    \label{paineis}
\end{figure}

\begin{figure}[H]
    \centering 
    \includegraphics[width=0.4\textwidth]{images/indicadores-globais.png} 
    \caption{Indicadores Globais - Versão Final} 
    \label{indicadores-globais}
\end{figure}

\begin{figure}[H]
    \centering 
    \includegraphics[width=0.4\textwidth]{images/paineis.png} 
    \caption{Indicadores Específicos - Versão Final} 
    \label{indicadores-especificos}
\end{figure}

Em relação aos testes de disponibilidade do sistema, registraram-se três ocasiões em que o servidor ficou indisponível. A primeira ocorreu devido a uma atualização automática do \textit{Windows 11}, situação que não pode ser contornada pelos mecanismos de inicialização configurados no Sistema Operacional, exigindo intervenção humana. O segundo caso foi provocado pelo modo de hibernação automática do \textit{Windows 11}. Entretanto, essa situação pôde ser resolvida mediante ajustes nas configurações de energia da máquina. Por fim o terceiro caso foi um erro inesperado do Docker, uma falha inesperada do Windows Subsystem for Linux (WSL) levou a interrupção do serviço, embora a reinicio da máquina foi suficiente para contornar o problema, ver Figura \ref{docker-crash}.

\begin{figure}[H]
    \centering 
    \includegraphics[width=0.4\textwidth]{images/docker-crash.png} 
    \caption{Interrupção por erro do WSL} 
    \label{docker-crash}
\end{figure}